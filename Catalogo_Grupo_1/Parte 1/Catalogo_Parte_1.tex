\documentclass[10pt]{article}
\usepackage[utf8]{inputenc}
\usepackage{amsmath,multicol,enumerate}
\usepackage{amsfonts}
\usepackage{multicol}

\usepackage{wrapfig}


%%%%% Formato REVISTA DE MATEMÁTICA: TEORÍA Y APLICACIONES%%%%%%%
\topmargin=-2cm\textheight=23cm\textwidth=19cm
\oddsidemargin=-1cm\evensidemargin=-1cm
%%%%%%%%%%%%%%%%%%%%%%%%%%%%%%%%%%%%%%%%%%%%%%%%%%%%%%%%%%%%%%%%%
\parskip=0.25cm
\parindent=0mm

\renewcommand{\figurename}{Figura}

\usepackage[dvips]{graphicx}
\usepackage{url}

%Utilizamos el paquete para incorporar graficos postcript


%\usepackage{amssymb}
\usepackage[psamsfonts]{amssymb} %paquetes para los simbolos matematicos
\usepackage{latexsym}

\usepackage{color}

\usepackage[framed]{matlab-prettifier}


\usepackage[T1]{fontenc}

\renewcommand{\lstlistingname}{C\'odigo}


\definecolor{mygray}{rgb}{0.4,0.4,0.4}
\definecolor{mygreen}{rgb}{0,0.8,0.6}
\definecolor{myorange}{rgb}{1.0,0.4,0}



%%%%% Para Insertar Lenguaje M (MATLAB/GNU OCTAVE)



\begin{document}

%Encabezado
Instituto Tecnológico de Costa Rica \hfill
CE-3102: Análisis Numéricos para Ingeniería\\
Ingeniería en Computadores \hfill 
Semestre: I - 2021\\




\begin{center}\textbf{\huge Catálogo Grupal de Algoritmos}  \end{center}

{\bf Integrantes: }
\begin{itemize}
\item Brayan Alfaro González - Carné 2019380074
\item Sebastián Alba Vives - Carné 2017097108 
\item Kevin Zeledón Salazar - Carné 2018076244 
\item Daniel Camacho González - Carné 2017114058
\end{itemize}


\section{Tema 1: Métodos para encontrar raíces en ecuaciones no lineales}

\subsection{Método de la Bisección}

\lstinputlisting[style=Matlab-editor, basicstyle=\mlttfamily\normalsize, caption={Método de la Bisección en Lenguaje M}]{biseccion.m}

\newpage

\subsection{Método de Newton Raphson}

\lstinputlisting[language=Python, basicstyle=\ttfamily\normalsize,morekeywords={as,__init__,MyClass}, keywordstyle=\color{violet}\ttfamily, frame=single, commentstyle=\color{gray}\ttfamily, showstringspaces=false, caption={Método de Newton Raphson en Python}, breaklines]{newton_raphson.py}

\newpage

\subsection{Método de la Secante}

\lstinputlisting[basicstyle=\normalsize\ttfamily\color{black}, commentstyle=\color{mygray}, frame=single, numbersep=5pt, numberstyle=\tiny\color{mygray}, keywordstyle=\color{mygreen}, showspaces=false, showstringspaces=false, stringstyle=\color{myorange}, tabsize=2, language=C++, caption={Método de la Secante en C++}, breaklines]{secante.cpp}

\newpage

\subsection{Método de la Falsa Posición}

\lstinputlisting[basicstyle=\normalsize\ttfamily\color{black}, commentstyle=\color{mygray}, frame=single, numbersep=5pt, numberstyle=\tiny\color{mygray}, keywordstyle=\color{mygreen}, showspaces=false, showstringspaces=false, stringstyle=\color{myorange}, tabsize=2, language=C++, caption={Método de la Falsa Posición en C++}, breaklines]{falsa_posicion.cpp}

\newpage

\subsection{Método del Punto Fijo}

\lstinputlisting[language=Python, basicstyle=\ttfamily\normalsize,morekeywords={as,__init__,MyClass}, keywordstyle=\color{violet}\ttfamily, frame=single, commentstyle=\color{gray}\ttfamily, showstringspaces=false, caption={Método del Punto Fijo en Python}]{punto_fijo.py}

\newpage

\subsection{Método de Muller}

\lstinputlisting[style=Matlab-editor, basicstyle=\mlttfamily\normalsize, caption={Método de Muller en Lenguaje M}]{muller.m}

\end{document}
