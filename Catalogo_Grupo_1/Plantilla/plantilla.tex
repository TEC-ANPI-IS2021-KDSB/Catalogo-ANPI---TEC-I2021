\documentclass[10pt]{article}
\usepackage[latin1]{inputenc}
\usepackage{amsmath,multicol,enumerate}
\usepackage{amsfonts}
\usepackage{multicol}

\usepackage{wrapfig}


%%%%% Formato REVISTA DE MATEM�TICA: TEOR�A Y APLICACIONES%%%%%%%
\topmargin=-2cm\textheight=23cm\textwidth=19cm
\oddsidemargin=-1cm\evensidemargin=-1cm
%%%%%%%%%%%%%%%%%%%%%%%%%%%%%%%%%%%%%%%%%%%%%%%%%%%%%%%%%%%%%%%%%
\parskip=0.25cm
\parindent=0mm

\renewcommand{\figurename}{Figura}

\usepackage[dvips]{graphicx}
\usepackage{url}

%Utilizamos el paquete para incorporar graficos postcript


%\usepackage{amssymb}
\usepackage[psamsfonts]{amssymb} %paquetes para los simbolos matematicos
\usepackage{latexsym}

\usepackage{color}

\usepackage[framed]{matlab-prettifier}


\usepackage[T1]{fontenc}

\renewcommand{\lstlistingname}{C\'odigo}


\definecolor{mygray}{rgb}{0.4,0.4,0.4}
\definecolor{mygreen}{rgb}{0,0.8,0.6}
\definecolor{myorange}{rgb}{1.0,0.4,0}



%%%%% Para Insertar Lenguaje M (MATLAB/GNU OCTAVE)



\begin{document}

%Encabezado
Instituto Tecnol�gico de Costa Rica \hfill CE-3102: An�lisis Num�ricos para Ingenier�a\\
Ingenier�a en Computadores \hfill Semestre: I - 2021\\




\begin{center}\textbf{\huge Cat�logo Grupal de Algoritmos}  \end{center}

{\bf Integrantes: }
\begin{itemize}
\item Juan Pablo Soto Quir�s - Carnet 2021123456
\item Lisbeth Lao Aguilar - Carnet 2021789045
\end{itemize}

\section{Tema 1: Nombre del Tema}

\subsection{M�todo 1: Aproximaci�n Num�rica del Inverso Multiplicativo}

\lstinputlisting[style=Matlab-editor, basicstyle=\mlttfamily\normalsize, caption={Lenguaje M.}]{ejemplo_lenguaje_m.m}

\newpage

\lstinputlisting[language=Python, basicstyle=\ttfamily\normalsize,morekeywords={as,__init__,MyClass}, keywordstyle=\color{violet}\ttfamily, frame=single, commentstyle=\color{gray}\ttfamily, showstringspaces=false, caption={Lenguaje Python.}]{ejemplo_lenguaje_python.py}

\newpage

\lstinputlisting[basicstyle=\normalsize\ttfamily\color{black}, commentstyle=\color{mygray}, frame=single, numbersep=5pt, numberstyle=\tiny\color{mygray}, keywordstyle=\color{mygreen}, showspaces=false, showstringspaces=false, stringstyle=\color{myorange}, tabsize=2, language=C++, caption={Lenguaje C++.}]{ejemplo_lenguaje_cpp.cpp}


\end{document}
